%%%%%%%%%%%%%%%%%%%%%%%%%%%%%%%%%%%%%%%%%
% Medium Length Professional CV
% LaTeX Template
% Version 2.0 (8/5/13)
%
% This template has been downloaded from:
% http://www.LaTeXTemplates.com
%
% Original author:
% Trey Hunner (http://www.treyhunner.com/)
%
% Important note:
% This template requires the resume.cls file to be in the same directory as the
% .tex file. The resume.cls file provides the resume style used for structuring the
% document.
%
%%%%%%%%%%%%%%%%%%%%%%%%%%%%%%%%%%%%%%%%%

%---------------------------------------------------------------------------------------
%	PACKAGES AND OTHER DOCUMENT CONFIGURATIONS
%---------------------------------------------------------------------------------------

\documentclass{resume} % Use the custom resume.cls style

\usepackage[left=0.75in,top=0.6in,right=0.75in,bottom=0.6in]{geometry} % Document margins
\usepackage[colorlinks=true,urlcolor=blue]{hyperref}

\name{Sergey Sokolov} 

\address{web-version: \href{http://teners.github.io}{\underline{teners.github.io}} \\ e-mail: \href{mailto:dmemory7@ya.ru}{\underline{dmemory7@ya.ru}}}
\address{telegram: \href{https://t.me/teners}{\underline{@teners}} \\ phone: +7~(900)~6-287-287}
\address{github: \href{https://github.com/teners}{\underline{github.com/teners}}}
\begin{document}

%---------------------------------------------------------------------------------------
%   GENERAL SECTION
%---------------------------------------------------------------------------------------
\begin{rSection}{General}
{\bf Date of Birth} \\
18 August 1996 \\
{\bf Goals} \\
C++ development, iOS\slash Android development, web development (backend primarily),
system administating, teaching IT (someday) \\
{\bf Field of Interests} \\
Software engineering, highload systems, Unix/Linux systems, open source, algorithms and
data structures, embedded systems, math \\
{\bf Personal Qualities} \\
Curious, flexible, ambitious, quick-learning, sociable, creative
\end{rSection}

%---------------------------------------------------------------------------------------
%	EDUCATION SECTION
%---------------------------------------------------------------------------------------

\begin{rSection}{Education}

{\bf State University of Aerospace Instrumentation, Saint-Petersburg} \hfill {\em 2013---Present} \\ 
B.S. Software Engineering (undergraduate)\\
GPA: 4.54

\end{rSection}

%----------------------------------------------------------------------------------------
%	WORK EXPERIENCE SECTION
%----------------------------------------------------------------------------------------

\begin{rSection}{Experience}
{\bf First Line Software student hackathon} \hfill {\em 24---25 October 2015} \\
``PS3D'' -- 3D particle system C++ library and GUI editor \\
{\bf Joint Advanced Student School -- JASS 2016} \hfill {\em 21---26 March 2016} \\
``Multimodel iNTeraction (Mint)'' -- UML diagram modelling tool for Android and iOS with real-time 
collaboration feature \\
{\bf GS Labs} \hfill {\em September 2016---Present} \\
C++ developer. Conditional Access Systems Libraries Department \\
{\bf First Line Software student hackathon} \hfill {\em 28---29 October 2016} \\
``WibeWM`` --- tiling window manager based on XCB library \\
{\bf Joint Advanced Student School -- JASS 2017} \hfill {\em 20---26 March 2017} \\
Web application backend for Zeiss \\

\end{rSection}

%----------------------------------------------------------------------------------------
%	TECHNICAL STRENGTHS SECTION
%----------------------------------------------------------------------------------------

\begin{rSection}{Technical Strengths}
%\begin{tabular}{ @{} >{\bfseries}l @{\hspace{6ex}} l }
{\bf Computer Languages \& Frameworks}\\ 
C++ (intermediate), Python (upper-intermediate), SQL (pre-intermediate), 
Java (beginner), HTML \& CSS (pre-intermediate), Javascript (pre-intermediate) \\
{\bf Tools} \\ 
bash, git, vim, gcc, cmake, LaTeX, pip, npm, redis, Sphinx \\
{\bf Languages} \\ 
Russian (native), English (upper-intermediate)
%\end{tabular}
\end{rSection}

%----------------------------------------------------------------------------------------
%	EXAMPLE SECTION
%----------------------------------------------------------------------------------------

%\begin{rSection}{Section Name}

%Section content\ldots

%\end{rSection}

%----------------------------------------------------------------------------------------

\end{document}

