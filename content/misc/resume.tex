% Original author:
% Trey Hunner (http://www.treyhunner.com/)


\documentclass{resume}

\usepackage[left=0.75in,top=0.6in,right=0.75in,bottom=0.6in]{geometry}
\usepackage[colorlinks=true,urlcolor=blue]{hyperref}

\name{Sergey Sokolov} 

\address{web-version: \href{https://teners.ru}{\underline{teners.ru}} \\ e-mail: \href{mailto:dmemory7@ya.ru}{\underline{dmemory7@ya.ru}}}
\address{telegram: \href{https://t.me/teners}{\underline{@teners}} \\ phone: +7~(900)~6-287-287}
\address{github: \href{https://github.com/teners}{\underline{github.com/teners}}}
\begin{document}

%   GENERAL
\begin{rSection}{General}
    {\bf Date of Birth} \\
        18 August 1996

    {\bf Goals} \\
        Web development, C\slash C++ system software development, teaching IT

    {\bf Field of Interests} \\
        Software engineering, highload, Unix\slash Linux systems, open source, programming languages, algorithms and data structures, embedded systems, STEM
\end{rSection}

%	EDUCATION
\begin{rSection}{Education}
    {\bf Saint Petersburg Electrotechnical University, Saint Petersburg} \hfill {\em 2017---2019} \\ 
    MS in Applied Mathematics and Informatics
    
    {\bf State University of Aerospace Instrumentation, Saint Petersburg} \hfill {\em 2013---2017} \\ 
    BS in Software Engineering \\
    Thesis topic: ``Effective LSM-tree implementation in Go'' \\
    GPA: 4.44
\end{rSection}

%	WORK EXPERIENCE
\begin{rSection}{Experience}
    {\bf Droice Labs} \hfill {\em since September 2017} \\
        Backend modules development in Python
    
    {\bf GS Labs} \hfill {\em September 2016---September 2017} \\
        C++ developer, Conditional Access Systems Libraries Department
\end{rSection}

\begin{rSection}{Teaching}
    {\bf Saint Petersburg Electrotechnical University} \hfill {\em since September 2017} \\
        Introduction to IT course seminar leader \\
        Introduction to NoSQL databases MOOC staff, course development and support

    {\bf Django Girls SPb} \hfill {\em 7 October 2017} \\
        Django Girls coach
\end{rSection}

\begin{rSection}{Schools \& Hackathons}        
    {\bf Joint Advanced Student School -- JASS 2017} \hfill {\em 20---26 March 2017} \\
        Web application backend for Zeiss
        
    {\bf First Line Software student hackathon} \hfill {\em 28---29 October 2016} \\
        ``WibeWM`` --- tiling window manager based on XCB library
        
    {\bf Joint Advanced Student School -- JASS 2016} \hfill {\em 21---26 March 2016} \\
        ``Multimodel iNTeraction (Mint)'' -- UML diagram modelling tool for Android and iOS with real-time collaboration feature

    {\bf First Line Software student hackathon} \hfill {\em 24---25 October 2015} \\
        ``PS3D'' -- 3D particle system C++ library and GUI editor
\end{rSection}

%	TECHNICAL STRENGTHS
\begin{rSection}{Technical Strengths}
    {\bf Keywords} \\ 
        Python, C++, SQL, Golang, Java, HTML \& CSS, Javascript, Coffeescript, Flask, Tornado, MongoDB, Docker, shell, git, vim, gcc, cmake, LaTeX, pip, pylint, npm, redis, Sphinx, IntelliJ IDEA based IDEs, Upsource

    {\bf Languages} \\ 
        Russian (native), English (advanced)
\end{rSection}

\end{document}

